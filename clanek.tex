\documentclass{csbulletin}
\usepackage{fontspec}
\usepackage{hologo}
\begin{document}
\title{\texttt{luavlna}: \texttt{vlna} pro formáty lua\TeX u}
\EnglishTitle{\texttt{luavlna}: \texttt{vlna} for lua\TeX\ formats}
\author{Michal Hoftich}
\maketitle
\begin{abstract}
Sem přijde abstrakt
\end{abstract}

Podle českých typografických norem by jednopísmenné předložky neměly stát na 
konci řádků. V \TeX u se tento problém řeší nahrazením mezery znakem \verb|~|,
který vkládá nezalomitelnou mezeru. Protože se na toto pravidlo snadno zapomíná,
existují různé pomůcky, které vkládají nezalomitelnou mezeru automaticky.

Klasickým nástrojem je program \verb|vlna|, který provádí nahrazení přímo na
úrovni zdrojového kódu. Později vznikly další projekty, které nemodifikují přímo
zdrojový kód, ale modifikují tok řetězců nebo tokenů při zpracování dokumentu 
\TeX em. Jedná se o \textsc{encxvlna} pro Enc\TeX a X\raisebox{.3ex}{Ǝ}Vlna pro \hologo{XeTeX}.
% Nedaří se mi použít záporné hodnoty pro \raisebox, je znak '-' aktivní?
Jejich výhodou je, 
\begin{summary}
	Anglický abstrakt

\end{summary}
\end{document}
